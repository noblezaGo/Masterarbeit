\chapter{Konzeption}

\section{�bersicht}

Ziel des Kapitels ist es, Konzepte aufzuzeigen, wie die in der Anforderungsanalyse erarbeiteten Anforderungen umgesetzt werden k�nnen.

\section{Erweiterung des bestehenden configurAIDERs um neue Features}

Der bestehende Code des configurAIDERs wird erweitert um neue Features, die bei der Bewertung der Anforderungen in Abschnitt \ref{sec:BewertungAnforderungen} mit (+) oder (++) bewertet wurden. Die an bereits bestehenden Features in Abschnitt \ref{sec:Funktionalit�t} kritisierten Punkte sollen verbessert werden.

Neue Funktionen:
\begin{itemize}
	\item Projektverwaltung: Es k�nnen Projekte angelegt werden, in denen unterschiedliche Modelle der GIGABOX projektiert sind und die zugeh�rigen Skripte+Include-Files
	\item Texteditor wird erweitert um:
	
	\begin{enumerate}
		\item M�glichkeit, letzte Schritte r�ckg�ngig zu machen
		\item Anzeige aller implementierten Funktionen zur Navigation im Skript
		\item Bei der Codeentwicklung unterst�tzende Funktionen
		\item Routing-Editor
	\end{enumerate}
	
	\item Frei andockbare Fenster
	\item Steuern und Beobachten von DIN, AIN, DOUT, SWITCH
	
\end{itemize}

Vorteil: 

\begin{itemize}
	\item Das Tool erh�lt damit einen erh�hten Funktionsumfang, als wichtig bewertete funktionale Anforderungen k�nnen umgesetzt werden.
	
\end{itemize} 

Nachteil: 
\begin{itemize}
	\item Nichtfunktionale Anforderungen wie Qualit�tskriterien an Software nach ISO9126 (Wartbarkeit, Usability, Effizienz, �bertragbarkeit, Zuverl�ssigkeit) k�nnen nicht erf�llt werden, sondern werden sich verschlechtern durch eine weitere Funktionserweiterung des Tools
	
	
	\item Viel Einarbeitung in bestehenden Code n�tig

\end{itemize}


\section{Neuentwicklung des configurAIDERs} 

Das Tool wird von Grund auf neu entwickelt. Es sollen die funktionalen und nichtfunktionalen Anforderungen aus Kapitel \ref{chp:Anforderungsanalyse} umgesetzt werden. Im Rahmen dieser Arbeit sollen vorerst nur die wichtigsten funktionalen Anforderungen umgesetzt werden (mit ++ bewertet). Allerdings sollen alle nichtfunktionalen Anforderungen erf�llt werden.

Funktionen:

\begin{itemize}
\item Projektverwaltung: Es k�nnen Projekte angelegt werden, in denen unterschiedliche Modelle der GIGABOX projektiert sind und die zugeh�rigen Skripte+Include-Files
\item Texteditor 
	\begin{itemize}
		\item mehrere Skripte lassen sich parallel �ber Tabs �ffnen
		\item Copy/Paste
		\item Suchen/Ersetzen
		\item M�glichkeit, letzte Schritte r�ckg�ngig zu machen
		\item Bei der Codeentwicklung unterst�tzende Funktionen. Ziel: Schnelles
und effektives Schreiben von fehlerfreiem Code
	\end{itemize}
	
\item Erkennung von GIGABOXEN, die mit dem PC verbunden sind
\item PAWN-Compiler
\item Beschreiben/Auslesen des Flash-Speichers der GIGABOX
\item Konsole
\item Bereitstellen von Dokus

\end{itemize}


Vorteil:
\begin{itemize}
	\item Tool wird einfacher erweiterbar als bisheriger Stand. Dies erm�glicht es, zuk�nftig einfacher neue Features zu integrieren.
	\item Besser wartbar: Bugs k�nnen in Zukunft besser behoben werden, da Verhalten des neuen Codes besser bekannt ist
	\item Usability kann verbessert werden
	\item Tool konsistent mit WPF realisiert unter aktueller .NET Version 4.6
	\item Keine lange Einarbeitung in alten Code n�tig
\end{itemize}

Nachteil:
\begin{itemize}
	\item In Entwurf der neuen Softwarearchitektur muss Zeit investiert werden
	\item Tool bietet keinen direkten Mehrwert in Form von h�herem Funktionsumfang 
	\item Insgesamt h�herer Aufwand um die gestellten funktionalen Anforderungen zu erreichen
\end{itemize}

