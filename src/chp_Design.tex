\chapter{Design}

\section{Entwurf der Benutzeroberfl�che}




Entwurf mit Pencil hier einf�gen

Erl�uterung des Entwurfes mit Bezugnahme auf ISO, die gute Usability definiert


	
\section{Entwurf der Softwarearchitektur}

\cite{EffSWArchitektur}

Eine Softwarearchitektur definiert die Komponenten eines Systems, beschreibt deren wesentliche Merkmale und charakterisiert die Beziehungen dieser Komponenten. Sie beschreibt den statischen Aufbau einer Software im Sinne eines Bauplans und den dynamischen Ablauf einer Software im Sinne eines Ablaufplans. 

Die in diesem Kapitel vorgestellte Software-Architektur des configurAIDERs wurde entworfen auf Basis der funktionalen und nichtfunktionalen Anforderungen an Software in Kapitel \ref{chp:Anforderungsanalyse}.

In Abbildung XX wird der configurAIDER in Kontext zu seiner Umgebung dargestellt. Der configurAIDER wird als Blackbox dargestellt und zeigt die Schnittstellen zu seiner Umgebung.  
\textcolor[rgb]{1,0,0}{Im Moment wird er nicht als Blackbox sondern Whitebox dargestellt
}

Abbildung 

Was ist Softwarearchitektur? IEEE 1471-2000


Architekturmuster: Layer-Architektur: MVVM
Bild mit MVVM Model, 3 Schichten

Architektur des configurAIDERS

Module:
Rahmenfenster  
Projektverwaltung
