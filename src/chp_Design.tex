\chapter{Design}

\section{Entwurf der Benutzeroberfl�che}




Entwurf mit Pencil hier einf�gen

Erl�uterung des Entwurfes mit Bezugnahme auf ISO, die gute Usability definiert


	
\section{Entwurf der Softwarearchitektur}

\cite{EffSWArchitektur}

Eine Softwarearchitektur definiert die Komponenten eines Systems, beschreibt deren wesentliche Merkmale und charakterisiert die Beziehungen dieser Komponenten. Sie beschreibt den statischen Aufbau einer Software im Sinne eines Bauplans und den dynamischen Ablauf einer Software im Sinne eines Ablaufplans. 

Die in diesem Kapitel vorgestellte Software-Architektur des configurAIDERs wurde entworfen auf Basis der funktionalen und nichtfunktionalen Anforderungen an Software in Kapitel \ref{chp:Anforderungsanalyse}. Wie im vorigen Kapitel "`Konzeption"' erl�utert, werden nur Funktionen umgesetzt, die als "`Essential"' oder "`Conditional"' priorisiert wurden.
\textcolor[rgb]{1,0,0}{Evtl. nochmal die Funktionen aufz�hlen (textuell oder als UseCaseDiagramm), die implementiert werden sollen}


\subsection{Entwurf der Kontextabgrenzung}

Die Kontextabgrenzung zeigt das System als Blackbox und stellt das System in Kontext zu seiner Umgebung dar. Abbildung \ref{fig:Kontextabgrenzung} zeigt die Kontextabgrenzung des configurAIDERs als UML Deployment Diagramm. Es werden alle den configurAIDER umgebenden Systeme dargestellt, die zur Erf�llung der in der Anforderungsanlayse hergeleiteten Use-Cases (nur die als "`Essential"' oder "`Conditional"' klassifizierten) ben�tigt werden.
Dazu z�hlt der Benutzer, eine GIGABOX und ein Datenspeicher. 

 
\begin{figure}[!htbp]
	\centering
		\includegraphics[width=\textwidth]{images/DeploymentDiagrammKontextabgrenzung.png}
	\caption{Darstellung der Kontextabgrenzung des Systems im UML-Deployment Diagramm}
	\label{fig:Kontextabgrenzung}
\end{figure}


\subsection{Entwurf der Laufzeitsicht}

Die Laufzeitsicht veranschaulicht, wie die Komponenten eines Systems zur Laufzeit zusammenarbeiten. 
F�r jede Komponente des configurAIDERs wurde eine Laufzeitsicht mithilfe von UML Sequenzdiagrammen erstellt. Dort kann f�r jede Komponente abgelesen werden, mit welchen anderen Komponenten sich kommuniziert im Rahmen der wichtigsten Use-Cases.

Abbildung XX zeigt die Laufzeitsicht des configurAIDERs als UML Sequenzdiagramm. F�r die wichtigsten Use-Cases werden die Kommunikationsabl�ufe zwischen den verschiedenen Komponenten dargestellt.

\begin{figure}[!htbp]
	\centering
		\includegraphics[width=\textwidth]{images/SequenzdiagrammLaufzeitsicht.png}
	\caption{Darstellung der Laufzeitsicht des Systems im UML Sequenzdiagramm}
	\label{fig:Laufzeitsicht}
\end{figure}



Was ist Softwarearchitektur? IEEE 1471-2000


Architekturmuster: Layer-Architektur: MVVM
Bild mit MVVM Model, 3 Schichten

Architektur des configurAIDERS

Module:
Rahmenfenster  
Projektverwaltung
