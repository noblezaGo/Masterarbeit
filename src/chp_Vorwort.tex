 
\chapter*{Danksagung} 
\addcontentsline{toc}{chapter}{Danksagung} % include to Inhaltsverzeichnis 

Diese Bachelorarbeit entstand in der Zeit von Juli bis November 2015 bei der Bosch Rexroth AG in Lohr am Main. Sie stellt den Abschluss meines Bachelorstudiums an der Hochschule Karlsruhe dar.

Ich bedanke mich f�r die Betreuung der Bachelorarbeit bei Prof. Dr. rer. nat. Klaus Wolfrum seitens der Hochschule Karlsruhe.

Besonderen Dank gilt meinen Firmenbetreuern Dipl.-Ing. Andr\'{e} Starke und Dipl.-Ing. Hagen Burchardt f�r die sehr gute Betreuung und die engagierte Beantwortung aller fachlicher Fragen w�hrend dieser Zeit.
Auch m�chte ich Dipl.-Ing. Georg Vetter danken f�r die gro�e Hilfe und Unterst�tzung bei der Softwareentwicklung.
Ferner gilt mein Dank allen Mitarbeitern der Abteilung DC-IA/EAI3 f�r die
Unterst�tzung und die stets freundliche und angenehme Zusammenarbeit.

In besonderem Ma�e bedanke ich mich bei meiner Familie, die dieses Studium erm�glichte und die mich immer unterst�tzt hat.

\chapter*{Abstract} 
\addcontentsline{toc}{chapter}{Abstract} % include to Inhaltsverzeichnis 
	
Intelligente Feldger�te mit integriertem Mikrocontroller lassen sich oftmals mithilfe von Parametern individuell an die Anforderungen einer Anlage in der Fertigungsindustrie anpassen. Dazu sind h�ufig herstellerspezifische Ger�tetools n�tig. Bei gro�en Projekten mit vielen Feldger�ten unterschiedlicher Hersteller wird die Parametrierung der Ger�te schnell zeitaufw�ndig und es wird schwer, den �berblick �ber die verschiedenen Ger�tetools zu behalten. 

Mit dem Ziel, dem Anwender die Ger�teparametrierung �ber die zugeh�rigen Ger�tetools zu vereinfachen, wurde die Aufrufschnittstelle "`Tool Calling Interface (TCI)"' von "`Profibus International (PI)"' spezifiziert. 

Die vorliegende Arbeit untersucht die Frage, wie TCI f�r das Engineering System IndraWorks der Firma Bosch Rexroth umgesetzt werden kann. Ziel ist eine prototypische Implementierung nach der TCI-Spezifikation Version 1.1 f�r PROFINET.
Dazu werden zwei Konzepte eines Prototyps vorgestellt, nach einer Abw�gung fiel die Entscheidung auf die Entwicklung einer Desktop-Applikation f�r IndraWorks.

Der realisierte Prototyp erm�glicht dem Anwender, �ber die Oberfl�che von IndraWorks  ein passendes Ger�tetool zu einem dort projektierten PROFINET-Ger�t aufzurufen. Dabei werden relevante Daten aus IndraWorks (u.a. Parameterdaten) an das Ger�tetool �bergeben.


