\chapter{Einleitung}

\section{Motivation}

Seit dem Aufkommen des CAN-Busses in den 1980er-Jahren nahm die Anzahl der verbauten Steuerger�te im PKW stetig zu. 
Mehr und mehr Sensoren und Aktoren werden eingesetzt und Infotainmentsysteme geh�ren inzwischen zur Standardaustattung eines Autos.
Aus diesen Gr�nden steigen die zu verarbeitenden Datenmengen.
Es werden leistungsf�hige Steuerger�te ben�tigt, die durch Bussysteme miteinander vernetzt sind und mit hohen Daten�bertragungsraten kommunizieren. Im Automotive-Bereich haben sich die Bussysteme CAN, LIN, MOST und FlexRay etabliert.

Zuk�nftig wird sich dieser Trend fortsetzen und das Datenaufkommen innerhalb von Fahrzeugen wird weiter steigen. Als Grund ist hier beispielsweise die Zunahme von Fahrassistenzsystemen zu nennen, die den Fahrer im Verkehr unterst�tzen.
Eine Antwort auf diese steigenden Anforderungen sind Busprotokolle wie das im Jahre 2012 ver�ffentlichte CAN FD-Protokoll, das eine bis zu 8-fach h�here Daten�bertragungsrate erm�glicht.

Obwohl die Fahrzeugssysteme komplexer werden, m�ssen die Hersteller aufgrund des Konkurrenzdrucks die Entwicklungszyklen f�r neue Fahrzeugmodelle verk�rzen und die Entwicklungs- und Herstellungskosten m�glichst gering halten. 
Auch bei der Entwicklung von Steuerger�ten muss auf diese Herausforderungen reagiert werden. Es werden M�glichkeiten ben�tigt, schnell und kosteng�nstig neue prototypische Funktionen zu entwickeln und diese innerhalb einer realen oder simulierten Fahrzeugumgebung zu testen.

Die Firma GIGATRONIK bietet mit ihren GIGABOX-Produkten daf�r eine L�sung. Die Universalsteuerger�te sind ausgestattet mit Kommunikationsschnittstellen, die im Automotive-Bereich g�ngig sind wie z. B. CAN und LIN. Zus�tzlich besitzen sie digitale und analoge Ein- und Ausg�nge zur Signalverarbeitung bzw. Steuerung und Regelung von Systemen.
F�r die Entwicklung von GIGABOX-Applikationen wird die integrierte Entwicklungsumgebung (IDE) \emph{configurAIDER} eingesetzt. Dort k�nnen Steuerger�tefunktionen auf einfache Weise programmiert werden unter Verwendung der Skriptsprache \emph{Pawn}.

Allerdings wird die aktuelle Produktpalette der GIGABOX nicht mehr den neuesten technischen Anforderungen gerecht, da 
diese beispielsweise kein CAN FD unterst�tzt.
Im Moment wird deshalb die GIGABOX FD entwickelt, welche die bisherige GIGABOX-Familie abl�sen soll. Im Vergleich zu den Vorg�ngermodellen besitzt diese eine CAN FD-Schnittstelle. Ebenfalls bietet sie die M�glichkeit, zus�tzliche Kommunikationstechnologien �ber Erweiterungsplatinen anzubinden. 

Eine moderne Desktop-Anwendungssoftware sollte dem Anwender eine attraktive und gut benutzbare Benutzeroberfl�che bieten.
Dadurch ergeben sich f�r den Nutzer Vorteile wie eine k�rzere Einarbeitungszeit in die Software, produktiveres Arbeiten sowie weniger Fehler bei der Bedienung. 
Ein gutes Beispiel f�r eine IDE mit einer gelungenen Benutzeroberfl�che ist \emph{Visual Studio} von Microsoft.  
F�r die neue GIGABOX FD soll deshalb eine moderne IDE zur Verf�gung stehen mit einer qualitativ hochwertigen Benutzeroberfl�che.



\section{Zielsetzung}	

Die Ziele der vorliegenden Arbeit sind:

\begin{itemize}
	\item Erarbeitung einer Anforderungsanalyse f�r eine IDE, die zur Erstellung von GIGABOX FD-Applikationen verwendet werden kann
	\item Umsetzung der Anforderungen auf Basis von standardisierten Softwareentwicklungsprozessen
	\item Erstellung einer Softwarearchitektur mithilfe von UML-Diagrammen
	\item Implementierung unter dem .NET Framework mit C\#. Zur Erstellung der grafischen Benutzeroberfl�che bietet sich die Verwendung von WPF an. 
\end{itemize}

